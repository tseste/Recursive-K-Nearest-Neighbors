% !TeX root = ../main.tex

A recommender system is a software that analyses the behavior of its users and
tries to recommend products relevant to their interests. Through the years a lot
of different recommendation techniques have been introduced. One of them is called
neighborhood-based collaborative filtering. This technique is one of the
earliest developed in the context of recommender systems. As in every
technique, collaborative filtering has its own advantages and disadvantages.
Its main disadvantage is that it needs lots of user ratings
before it can form enough relations to understand users interests and
recommend products.

In this thesis, a recursive approach will be introduced that tries to
overcome the limitations of the conventional neighborhood-based collaborative
filtering and generate more predictions, resulting in
better recommendations. As a case study the recursive nearest neighbors algorithm
was evaluated in the Epinions data set, with a variety of similarity metrics
such as Cosine Similarity, Pearson Correlation Coefficient, etc and with
different error metrics RMSE, MAE, MAUE, RMSUE in order to test its performance.
The results showed that there has been a significant increase in the number of rating
predictions and at the same time the inclusion of these new rating predictions
in the error metrics resulted in only a slight increase in the total error of the model.\\\\
\justify
\textbf{Keywords:} Recommender Systems, Collaborative Filtering,
                   Nearest Neighbors, Recursive method, \\Epinions dataset
