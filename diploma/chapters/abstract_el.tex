% !TeX spellcheck = el-GR
% !TeX root = ../main.tex

Ένα σύστημα συστάσεων είναι ένα λογισμικό που αναλύει τη
συμπεριφορά των χρηστών του και προσπαθεί να συστήσει προϊόντα
σχετικά με τα ενδιαφέροντα τους. Κατά το πέρασμα των χρόνων
εισήχθησαν πολλές διαφορετικές τεχνικές σύστασης. Μια από αυτές ονομάζεται
συνεργατικό φιλτράρισμα βάσει γειτονιάς. Αυτή η τεχνική είναι μία
από τις πρώτες που αναπτύχθηκαν στο πλαίσιο των συστημάτων
συστάσεων. Όπως σε κάθε τεχνική, έτσι και το συνεργατικό φιλτράρισμα έχει τα
δικά του πλεονεκτήματα και μειονεκτήματα. Το κύριο μειονέκτημα του
είναι ότι χρειάζεται πολλές βαθμολογίες από τους χρήστες
πριν να μπορέσει να δημιουργήσει αρκετές διασυνδέσεις
για να κατανοήσει τα ενδιαφέροντα των χρηστών και να προτείνει
προϊόντα.

Σε αυτή τη διπλωματική εργασία θα εισαχθεί μια αναδρομική προσέγγιση που
θα προσπαθήσει να ξεπεράσει τους περιορισμούς του συνηθισμένου
γειτονικού φιλτραρίσματος και να δημιουργήσει
περισσότερες προβλέψεις, που θα έχει ως αποτέλεσμα καλύτερες συστάσεις.
Ως μελέτη περίπτωσης, ο αλγόριθμος αναδρομικών πλησιέστερων γειτόνων
αξιολογήθηκε στο σύνολο δεδομένων Epinions με μια ποικιλία μετρήσεων ομοιότητας
όπως Ομοιότητα συνημιτόνου, Συντελεστής Συσχέτισης Pearson κ.λπ. και με διαφορετικές
μετρικές σφαλμάτων RMSE, MAE, MAUE, RMSUE με σκοπό να δοκιμασθεί η απόδοσή του.
Τα αποτελέσματα έδειξαν ότι υπήρξε σημαντική αύξηση του αριθμού των προβλέψεων αξιολόγησης
και ταυτόχρονα η συμπερίληψη αυτών των νέων προβλέψεων αξιολόγησης
στις μετρικές σφάλματος έδειξε ότι το συνολικό σφάλμα του μοντέλου αυξήθηκε ελάχιστα.\\
\justify
\textbf{Λέξεις Κλειδιά:} Συστήματα Υποδείξεων, Συνεργατικό Φιλτράρισμα,
Πλησιέστεροι Γείτονες, \\Αναδρομική Μέθοδος, σύνολο δεδομένων Epinions
